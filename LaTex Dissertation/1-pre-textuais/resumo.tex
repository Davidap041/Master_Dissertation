Atualmente o controle e modelagem de sistemas industriais, envolvem um controle supervisório com processos a eventos discretos e também o controle entre os agentes envolvidos na planta para realizar determinado processo.
O presente trabalho visa desenvolver uma técnica de controle e modelagem para sistema multiagentes, em que a modelagem é feita através da rede de Petri e o controle através do controle por consenso.
A modelagem dos eventos ao longo do sistema são feitos através da rede de petri, com transições e lugares. O controle é feito através do algorítimo de consenso de multiagentes em que é feita a sincronia dos agentes da planta para diferentes pontos de operação alternados pela rede de petri.
A técnica de controle e modelagem é apresentada através da aplicação em um cenário de sincronia e formação de autômatos em uma trajetória definida, em que a ordem e a forma de organização dos autômatos é alterada ao longo de eventos modelados pela rede de petri.
É possível perceber que a técnica de utilização conjunta da modelagem em redes de petri com o controle por consenso possui uma boa performance e abstração em termos de modelagem de um sistema envolvendo muitos agentes com diferentes eventos, assim como o uso de controle por consenso auxilia de forma eficiente a sincronia entre os diversos agentes industriais.

% Separe as palavras-chave por ponto
\palavraschave{Controle por Consenso; Redes de Petri; Controle multiagente; Plantas industrias, Sistemas Híbridos.}