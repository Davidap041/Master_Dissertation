O controle e modelagem de sistemas industriais envolvem a implementação de um sistema supervisório para processos de eventos discretos, bem como o controle entre os agentes na planta para executar processos específicos. Nesse cenário, os componentes industriais e os controladores locais não operam isoladamente no processo de manufatura, tornando essencial a otimização das referências e a sincronia por meio do controle cooperativo. Essa abordagem busca alcançar uma curva otimizada de referência para os controladores locais, simultaneamente evitando colisões entre os agentes. Na modelagem do sistema industrial, a necessidade de uma representação intuitiva e detalhada é atendida por meio das Redes de Petri Colorida. As Redes de Petri Colorida apresentam não apenas uma representação gráfica e hierárquica orientada a eventos, mas também proporcionam uma linguagem que facilita a comunicação com diferentes tipos de mensagens ao longo da rede.
O objetivo deste trabalho é desenvolver uma abordagem integrada de modelagem e simulação, utilizando Redes de Petri Colorida e estratégias de Controle Cooperativo, para uma planta industrial.
A modelagem dos eventos ao longo do sistema é realizada por meio da Rede de Petri, utilizando transições e lugares. O controle é implementado através do algoritmo de consenso de multiagentes, garantindo a sincronização dos agentes da planta para diferentes referências e etapas de operação na Rede de Petri.
A técnica de controle e modelagem é apresentada por meio da aplicação em um cenário de sincronia e formação de autômatos em uma trajetória definida, em que a ordem e a forma de organização dos autômatos são alteradas ao longo de eventos modelados pela Rede de Petri.
Verificou-se que a utilização conjunta da modelagem em redes de petri com o controle por consenso possui uma boa performance e abstração em termos de modelagem de um sistema envolvendo muitos agentes com diferentes eventos, assim como o uso de controle por consenso auxilia de forma eficiente a sincronia entre os diversos agentes industriais.

\palavraschave{Controle por Consenso; Redes de Petri; Controle multiagente; Plantas industrias, Sistemas Híbridos.}
