The control and modeling of industrial systems involve the implementation of a supervisory system for discrete event processes, as well as control among the agents in the plant to execute specific processes. In this scenario, industrial components and local controllers do not operate independently in the manufacturing process, making it essential to optimize references and synchronize through cooperative control. This approach aims to achieve an optimized reference curve for local controllers while simultaneously avoiding collisions between agents. In the modeling of the industrial system, the need for an intuitive and detailed representation is addressed through Colored Petri Nets. Colored Petri Nets not only provide a graphical and hierarchical event-oriented representation but also offer a language that facilitates communication with different types of messages throughout the network.
The objective of this work is to develop an integrated approach to modeling and simulation, using Colored Petri Nets and Cooperative Control strategies, for an industrial plant.
The modeling of events throughout the system is carried out through the Petri Net, using transitions and places. Control is implemented through the multi-agent consensus algorithm, ensuring the synchronization of plant agents for different references and operation stages in the Petri Net.
The control and modeling technique is presented through application in a scenario of synchronization and formation of automata along a defined trajectory, where the order and organization of automata are altered over events modeled by the Petri Net.
It was found that the combined use of Petri Net modeling with consensus control performs well and abstracts the modeling of a system involving many agents with different events. Additionally, the use of consensus control efficiently aids in synchronizing various industrial agents.

\keywords{Consensus Control; Petri Nets; Multi-agent Control; Industrial Plants; Hybrid Systems.}