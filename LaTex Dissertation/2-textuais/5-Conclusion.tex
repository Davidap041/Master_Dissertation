\chapter{Conclusão}
\label{chap:conclusion}
Este trabalho realizou uma abordagem integrada de modelagem e controle em sistemas industriais, utilizando Redes de Petri Colorida e estratégias de controle cooperativo. 
O método proposto de modelagem e controle foi aplicado a um problema prático envolvendo uma planta industrial, abordando a sincronia e formação de autômatos ao longo de trajetórias definidas. Eventos modelados pela rede de Petri foram utilizados para alterar a formação e os pontos de sincronia do grupo de autômatos. 

A implementação de um modelo detalhado da planta industrial por meio de Redes de Petri Colorida mostrou-se eficaz e intuitiva, oferecendo uma visualização hierárquica e orientada a eventos, além de facilitar a comunicação entre os componentes da planta. Assim, as Redes de Petri Colorida destacaram-se como uma ferramenta fundamental na modelagem precisa e abstrata de sistemas complexos, contribuindo para uma compreensão aprofundada e eficiente do funcionamento da planta industrial. 

O método de abordagem integrada revelou-se viável e eficiente, especialmente em aplicações com muitos agentes, já que o controle por consenso proporciona uma implementação simples, sem grande demanda matemática, fornecendo a sincronia necessária para a aplicação de uma formação ordenada do grupo de autômatos. 

No ponto de vista de robustez e adaptabilidade do sistema, observou-se que cada agente respeita as limitações dos agentes vizinhos, seja ela de posição, de velocidade, evitando assim colisões. Independente da mudança da dinâmica de um agente todo o sistema tem sua dinâmica adaptada, trazendo assim uma sincronia entre os diferentes agentes com diferentes comportamentos ao longo do sistema.

A principal contribuição  deste trabalho consistiu no uso da técnica integrada de modelagem e controle que diminui o processamento local em cada agente, deixando assim as lógicas de processamento centralizadas em um sistema supervisório modelado via rede de petri, além de uma lógica de controle de baixo custo computacional. Todavia é necessário ainda haver uma ótima comunicação entre os agentes, pois a base do controle é dada pela sincronia entre os estados do agente vizinho.
