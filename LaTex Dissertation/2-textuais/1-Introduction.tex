\chapter{introdução}
\label{chap:introduction}

%Para começar a usar este \textit{template}, na plataforma \textit{ShareLatex}, vá nas opções (três barras vermelhas horizontais) no canto esquerdo superior da tela e clique em "Copiar Projeto" e dê um novo nome para o projeto. 



%Testando o símbolo $\symE$

%\lipsum[5]  % Simulador de texto, ou seja, é um gerador de lero-lero.

%	\begin{alineas}
%		\item Lorem ipsum dolor sit amet, consectetur adipiscing elit. Nunc dictum sed tortor nec viverra.
%		\item Praesent vitae nulla varius, pulvinar quam at, dapibus nisi. Aenean in commodo tellus. Mauris molestie est sed justo malesuada, quis feugiat tellus venenatis.
%		\item Praesent quis erat eleifend, lacinia turpis in, tristique tellus. Nunc dictum sed tortor nec viverra.
%		\item Mauris facilisis odio eu ornare tempor. Nunc dictum sed tortor nec viverra.
%		\item Curabitur convallis odio at eros consequat pretium.
%	\end{alineas}



Um sistema industrial é composto por sensores, atuadores, sinalizadores, controladores entre outros componentes voltados para a realização de determinada cadeia de processos dentro de uma linha de produção. Tal que para realizar determinado processo é necessário uma sincronia entre diversos equipamentos, sensores e atuadores ao longo da planta industrial. Além do desafio de sincronizar uma gama de processos, os sistemas modernos possuem a necessidade de adaptar-se a novas variações e configurações, abrindo espaço para máquinas e sistemas com programação mais robusta e reconfigurável. 

Dado este desafio, as redes de Petri coloridas se oferecem como uma ótima ferramenta de modelagem para os sistemas modernos de manufatura em linha de produção em que há um aumento da versatilidade e flexibilidade da estrutura e também a necessidade de uma programação com alto nível de abstração. \cite{framework}

De acordo com \cite{discrete}, as redes de Petri têm sido consideradas com um modelo adequado para um controle supervisório com o objetivo de abranger uma grande classe de problemas e explorar a análise algébrica necessária para otimização. Tratando-se também da análise para a planta não alcançar determinadas marcações indesejadas;

As redes de Petri também são uma ferramenta de modelagem inicial para o algoritmo de programação com ferramentas intrínsecas que analisam o algoritmo para evitar que o sistema entre em exceções,\cite{embeddedOO}

Para modelagem e controle desse sistema composto por vários agentes, escolheu-se a abordagem por redes de Petri coloridas. As redes de Petri  coloridas são uma ferramenta gráfica e matemática que se adaptam bem a um grande numero de aplicações, tais como protocolos de comunicação, controle de oficinas de fabricação. 

A complexidade dos sistemas, em particular o de fabricação automatizada, leva a uma decomposição de vários níveis de controle, tais como planejamento, escalonamento, coordenação global, coordenação de sub-sistemas e controle direto (autômatos programáveis conectados aos sensores e aos atuadores). \cite{vallete}

A utilização de redes de petri como camada de abstração para tomada de decisões, escolha de estratégias diante dos problemas e organização dos agentes para seguir um determinado plano foi trabalhada como framework por \cite{Ebadi2010}, em que todo o processo de mais alto nível de decisão foi modelado para a rede de petri, nesse presente trabalho além da abstração pela rede de petri como tomada de decisão também é apresentando o algoritmo de consenso como solução do problema de controle e cooperação entre os agentes dada uma tomada de decisão de organização dos grupos.



% De outra forma, abordagens mais simplificadas e de fácil ...

% Em se tratando de soluções numéricas ...

% Um grande desafio presente ...

\begin{comment}   

\section{Objetivos}
\subsection{Objetivos gerais}
% O presente trabalho dedica-se ....


\subsection{Objetivos específicos}
% Os objetivos específicos deste trabalho podem ser resumidos em
\begin{itemize}
    \item Modelagem ... ;
    \item Abordagem ...;
    \item Implementação ....; 
\end{itemize}

\end{comment}

\begin{comment}   
\section{Justificativa}
% Uma vez que ....
\end{comment}

\begin{comment}   
 \section{Motivação}
A abordagem .... 
\end{comment}

\begin{comment}   
\section{Produção científica}
Ao longo do desenvolvimento desta dissertação, foram publicados ou submetidos a
congressos ou periódicos os seguintes artigos:
\begin{itemize}
    \item PAIVA, Davi Alexandre et al. A simple procedure for modeling and identification of a test bench 4-DOF manipulator. In: Congresso Brasileiro de Automática-CBA. 2020.
\end{itemize}
\end{comment}

\section{Organização do trabalho}
Para a estruturação do presente trabalho, adota-se a seguinte metodologia de estudo

\begin{enumerate}
\item 
\textbf{Introdução:} Este capítulo contém as premissas básicas de estudo e evolução dos temas recorrentes na área de Controle Multiagentes. Ainda incluem-se os princípios básicos de apresentação do projeto, tais como os objetivos, a justificativa e a motivação do estudo.

\item \textbf{Sistema Multiagentes: } Partindo-se do princípio mais básico relacionado a modelagem de sistemas multiagentes. Assim, definem-se as representações matemáticas e gráficas de um sistema multiagente assim como técnicas de controle Cooperativo. Por fim são repassadas as principais técnicas de modelagem, representação em grafos e controle de sistemas multiagentes. 
    
\item \textbf{Redes de Petri: } 
    
\item \textbf{Simulação:} 
    
\item \textbf{Conclusão:} Por fim, este último capítulo trata das considerações gerais sobre os conceitos apresentados e uma discussão crítica acerca dos resultados de simulação.

\end{enumerate}

	
