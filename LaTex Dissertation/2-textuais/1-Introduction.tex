\chapter{introdução}
\label{chap:introduction}
 Um sistema industrial é composto por sensores, atuadores, sinalizadores, controladores entre outros componentes voltados para a realização de determinada cadeia de processos dentro de uma linha de produção. Tal que para realizar determinado processo é necessário uma sincronia entre diversos equipamentos, sensores e atuadores ao longo da planta industrial. 
 
Além do desafio de coordenar uma gama de processos, os sistemas industriais inteligentes integrados possuem a necessidade de adaptar-se a novas variações e configurações, abrindo espaço para máquinas e sistemas com programação mais robusta e reconfigurável. 

Visando abordar a problemática do aprimoramento da coordenação entre os diversos componentes e controladores em sistemas industriais, a eficiência dos processos e a otimização de recursos representam alguns dos principais desafios enfrentados pela indústria, sendo alvo de estratégias contínuas de melhoria.

Dado este desafio, as redes de Petri coloridas se oferecem como uma ótima ferramenta de modelagem para os sistemas modernos de manufatura em linha de produção, em que há um aumento da versatilidade e flexibilidade da estrutura e também a necessidade de uma programação com alto nível de abstração.  \cite{framework}

A complexidade dos sistemas, em particular o de fabricação automatizada, leva a uma decomposição de vários níveis de controle, tais como planejamento, escalonamento, coordenação global, coordenação de sub-sistemas e controle direto (autômatos programáveis conectados aos sensores e aos atuadores).  \cite{vallete}

No contexto do controle é aplicada a estratégia de controle cooperativo em sistemas multi-agentes para a coordenação de vários agentes. Essa abordagem busca alcançar uma curva otimizada de mudança de referência para os controladores locais, enquanto simultaneamente evita colisões entre os agentes. Adicionalmente, a estratégia promove robustez e adaptabilidade diante de possíveis travamentos, desgaste e mudanças de comportamento entre os componentes da planta.

Nesse sentido, o objetivo desta pesquisa é analisar e desenvolver uma abordagem integrada de modelagem e simulação, utilizando Redes de Petri Colorida e estratégias de Controle Cooperativo, aplicadas a uma planta industrial. 
De forma mais específica, buscou-se: realizar um estudo sobre o estado atual do controle cooperativo em sistemas multiagentes, com ênfase na modelagem por redes de Petri; propor um modelo detalhado de uma planta industrial utilizando redes de Petri colorida; aplicar estratégias de controle cooperativo para promover consenso entre os diferentes componentes na planta industrial; e integrar o Controle Cooperativo com a simulação de um modelo de planta industrial por meio de Redes de Petri Colorida, buscando uma análise mais abrangente e eficaz das interações e comportamentos do sistema. 

A metodologia empregada envolve a integração da modelagem dos eventos ao longo do sistema, realizada por meio da rede de Petri, que utiliza transições e lugares, com o controle implementado através do algoritmo de consenso de multiagentes. A técnica de controle e modelagem é apresentada através da aplicação em um cenário de sincronia e formação de autômatos em uma trajetória definida, em que a ordem e a forma de organização dos autômatos é alterada ao longo de eventos modelados pela rede de Petri.

%%%%

%De acordo com \cite{discrete}, as redes de Petri têm sido consideradas com um modelo adequado para um controle supervisório com o objetivo de abranger uma grande classe de problemas e explorar a análise algébrica necessária para otimização. Tratando-se também da análise para a planta não alcançar determinadas marcações indesejadas;

%As redes de Petri também são uma ferramenta de modelagem inicial para o algoritmo de programação com ferramentas intrínsecas que analisam o algoritmo para evitar que o sistema entre em exceções, \cite{embeddedOO}.
%A utilização de redes de petri como camada de abstração para tomada de decisões, escolha de estratégias diante dos problemas e organização dos agentes para seguir um determinado plano foi trabalhada como framework por \cite{Ebadi2010}, em que todo o processo de mais alto nível de decisão foi modelado para a rede de petri, nesse presente trabalho além da abstração pela rede de petri como tomada de decisão também é apresentando o algoritmo de consenso como solução do problema de controle e cooperação entre os agentes dada uma tomada de decisão de organização dos grupos.

% De outra forma, abordagens mais simplificadas e de fácil ...

% Em se tratando de soluções numéricas ...

% Um grande desafio presente ...
\section{Justificativa}
Na perspectiva da Engenharia de Controle e Automação, a integração do ciclo de modelagem, controle e simulação dos sistemas é essencial para orientar o planejamento e a programação dos controladores industriais, bem como para monitoramento no sistema supervisório. Através da modelagem e controle é possível obter impactos significativos na eficiência operacional do sistema, prevenção de falhas e bloqueios, otimização do uso de recursos, eficiência energética e sincronização eficaz entre os diversos componentes da planta. 

De maneira geral, em sistemas de automação, o processo de manufatura não se realiza apenas por meio de componentes isolados, mas sim por meio de um conjunto em sincronia, executando etapas específicas de um processo. Assim, como controle de um sistema industrial não se limita apenas a controladores locais em cada componente de sensores e atuadores, mas também a otimização do acionamento e referência desses controladores locais visando a cooperação entre diversos agentes de uma planta para realizar uma sequência de etapas e processamento de peças e produtos.

Assim como o controle supervisório, híbrido, entre outros, apresentados por \cite{cassandras} o controle cooperativo é aplicado a um controle hierarquicamente de mais alto nível em relação aos controladores locais, emprega um algoritmo de consenso entre os diversos agentes envolvidos. Essa estratégia de controle alcança a curva otimizada de mudança de referência para os controladores locais, ao mesmo tempo em que evita colisões entre os  agentes e promove mais robustez e adaptabilidade a travamentos, desgaste e mudanças de comportamento entre os componentes da planta.

% adicionar um parágrafo de justifica que explique o uso de redes de petri colorida para a modelagem do sistema industrual
A escolha da abordagem com Redes de Petri Colorida para a modelagem do sistema industrial é respaldada pela capacidade única dessa metodologia em representar visualmente as interações complexas entre os diferentes componentes de um sistema. \cite{jensen} atribui que as Redes de Petri oferecem não apenas uma representação orientada a eventos, mas também uma linguagem que possibilita a modelagem de diferentes tipos de mensagens ao longo da rede.  Além disso, permitem a adoção de modelos hierárquicos, como os exemplos top-down e bottom-up, proporcionando uma análise, compreensão e nível de detalhamento mais preciso e abrangente. Essas características contribuem significativamente para uma representação visual completa e informativa do sistema industrial em questão.

 \section{Motivação}
O Programa de Pós-Graduação em Engenharia Elétrica da Universidade Federal do Ceará tem concentrado seus estudos na área de controle e automação industrial, explorando diversas abordagens de controladores e modelagens para esse propósito.

A proposta de elaborar um modelo detalhado de uma planta industrial utilizando Redes de Petri Colorida, além de aplicar estratégias de controle cooperativo para alcançar consenso entre os diferentes componentes e integrar o Controle Cooperativo com a simulação do modelo, representa uma abordagem inovadora e vantajosa para a automação industrial.

O controle cooperativo, ao promover a colaboração entre os diversos elementos da planta, busca otimizar a eficiência operacional, melhorar a resposta a eventos imprevistos e aumentar a flexibilidade do sistema. A busca por consenso entre os componentes é crucial para evitar conflitos e assegurar um funcionamento harmonioso.

A escolha das Redes de Petri Colorida como base para o modelo oferece uma representação visual clara e precisa da dinâmica do sistema, permitindo uma análise detalhada dos processos e comportamentos da planta. Essa abordagem facilita a identificação de pontos críticos, a avaliação de cenários complexos e a implementação efetiva de estratégias de controle.

Portanto, a combinação do Controle Cooperativo com Redes de Petri Colorida oferece uma solução robusta e eficiente para enfrentar os desafios da automação industrial, proporcionando um ambiente mais adaptável, resiliente e capaz de lidar com as demandas dinâmicas e imprevisíveis encontradas em ambientes industriais complexos.


\section{Estado da arte}
O desenvolvimento de um sistema robótico multiagente SMART, em que o conceito de agentes é aplicado tanto para as entidades de hardware quanto a de software.  O agentes de hardware cooperam fortemente com os agentes de software que são classificados como processamento de imagem, comunicação, gerenciamento de tarefas, tomadas de decisão, planejamento trajetória. Para a modelagem, controle e avaliação das tarefas cooperativas entre os agentes é utilizadas um tipo de Rede de Petri chamado Work-Flow Petri Nets. O principal aspecto de um sistema SMART é o problema de coordenação, que garante que a decisão tomada por cada agente coopere com o grupo inteiro, de modo que em um sistema multiagente o ponto crucial é a arquitetura de comunicação entre os agentes. \cite{smart2013}

A estratégia de controle de sistemas multiagentes através da abordagem de transmissão de token para coordenar a distribuição de tarefas para um grupo de robôs. \cite{token2006} 

\section{Objetivos}
\subsection{Objetivos gerais}
 O presente trabalho dedica-se a desenvolver uma abordagem integrada de modelagem e simulação utilizando Redes de Petri Colorida e estratégias de Controle Cooperativo de uma planta industrial.

\subsection{Objetivos específicos}
\begin{itemize}[]
    \item {Realizar um estudo acerca do estado da arte no campo do controle cooperativo em sistemas multiagentes com modelagem em redes de Petri.}
    
    \item {Propor um modelo detalhado de uma planta industrial utilizando rede de Petri colorida.}
    
    \item {Aplicar estratégias de controle cooperativo para um consenso entre os diferentes componentes na planta industrial.}

    \item {Integrar  o  Controle Cooperativo com a  simulação de um modelo de uma planta industrial utilizando Redes de Petri Colorida.} 
\end{itemize}


\section{Produção Científica}
Durante o desenvolvimento desta dissertação, foram publicados ou submetidos a congressos ou periódicos os seguintes artigos:

\begin{itemize}
    \item \textbf{PAIVA, Davi Alexandre; VASCONCELOS, Felipe José de Sousa; FILGUEIRAS, Iury de Amorim Gaspar; CORREIA, Wilkley Bezerra}. A simple procedure for modeling and identification of a test bench 4-dof manipulator. In: SOCIEDADE BRASILEIRA DE AUTOMÁTICA (SBA) AND GALOá SCIENCE.
    Congresso Brasileiro de Automática. Online: Galoá, 2020. (CBA2020, 1), p. 1050. ISSN
    2525-8311. Aparece nas coleções: DEEL - Trabalhos apresentados em eventos. Disponível em:
    http://www.repositorio.ufc.br/handle/riufc/65012.
    %\cite{paiva2020}

    \item \textbf{VASCONCELOS, F. J. S.; LEITE, G. C.; NETO, G. B. F.; CORREIA, W. B.; AGUIAR, V. P. B.; PAIVA, D. A.} ANFIS Identification Applied to a Reservoir Level Liquid System. 2021 9th International Conference on Control, Mechatronics and Automation (ICCMA), Belval, Luxembourg, 2021, pp. 135-140, doi: 10.1109/ICCMA54375.2021.9646226.
    %\cite{vasconcelos2021}
\end{itemize}



\section{Organização do trabalho}
O texto está organizado em capítulos, permitindo uma abordagem sequencial lógica
dos temas, conforme a seguinte estrutura:
\begin{itemize}
\item \textbf{Capítulo 1:} Este capítulo contém as premissas básicas de estudo e evolução dos temas recorrentes na área de Controle Multiagentes. Ainda incluem-se os princípios básicos de apresentação do projeto, tais como os objetivos, a justificativa e a motivação do estudo.

\item \textbf{Capítulo 2: } Partindo-se do princípio mais básico relacionado a modelagem de sistemas multiagentes. Assim, definem-se as representações matemáticas e gráficas de um sistema multiagente assim como técnicas de controle Cooperativo. São abordadas as principais estratégias de modelagem, representação em grafos e controle de sistemas multiagentes, com um enfoque especial na interação entre agentes.
    
\item \textbf{Capítulo 3: } Este segmento explora o uso de Redes de Petri como uma ferramenta fundamental na modelagem de sistemas a eventos discretos e, especificamente, em contextos de controle cooperativo em sistemas multiagentes. Serão apresentados conceitos essenciais, representações gráficas e aplicações práticas das Redes de Petri Colorida.
    
\item \textbf{Capítulo 4:}  Neste tópico, abordaremos a importância da simulação na validação e análise de sistemas multiagentes modelados por Redes de Petri. 
    
\item \textbf{Capítulo 5:} Por fim, este último capítulo trata das considerações gerais sobre os conceitos apresentados e uma discussão crítica acerca dos resultados de simulação.
\end{itemize}


	
% Introdução confusa, falta os links e orgaização melhor.

% Finalizar o parágrafo com as próprias palavras, 
% Avançar com a organização do trabalho, deixar mais específico. 
% Explicar metodologia  na Introdução